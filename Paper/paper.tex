\documentclass[preprint,review,1p]{elsarticle} %review=doublespace preprint=single 5p=2 column
%%% Begin My package additions %%%%%%%%%%%%%%%%%%%

\usepackage[hyphens]{url}

  \journal{Journal Name} % Sets Journal name

\usepackage{graphicx}
%%%%%%%%%%%%%%%% end my additions to header

\usepackage[T1]{fontenc}
\usepackage{lmodern}
\usepackage{amssymb,amsmath}
% TODO: Currently lineno needs to be loaded after amsmath because of conflict
% https://github.com/latex-lineno/lineno/issues/5
\usepackage{lineno} % add
\usepackage{ifxetex,ifluatex}
\usepackage{fixltx2e} % provides \textsubscript
% use upquote if available, for straight quotes in verbatim environments
\IfFileExists{upquote.sty}{\usepackage{upquote}}{}
\ifnum 0\ifxetex 1\fi\ifluatex 1\fi=0 % if pdftex
  \usepackage[utf8]{inputenc}
\else % if luatex or xelatex
  \usepackage{fontspec}
  \ifxetex
    \usepackage{xltxtra,xunicode}
  \fi
  \defaultfontfeatures{Mapping=tex-text,Scale=MatchLowercase}
  \newcommand{\euro}{€}
\fi
% use microtype if available
\IfFileExists{microtype.sty}{\usepackage{microtype}}{}
\usepackage[]{natbib}
\bibliographystyle{plainnat}

\ifxetex
  \usepackage[setpagesize=false, % page size defined by xetex
              unicode=false, % unicode breaks when used with xetex
              xetex]{hyperref}
\else
  \usepackage[unicode=true]{hyperref}
\fi
\hypersetup{breaklinks=true,
            bookmarks=true,
            pdfauthor={},
            pdftitle={Agroforestry SmartAgroforst: Poplar biomass modelling},
            colorlinks=false,
            urlcolor=blue,
            linkcolor=magenta,
            pdfborder={0 0 0}}

\setcounter{secnumdepth}{0}
% Pandoc toggle for numbering sections (defaults to be off)
\setcounter{secnumdepth}{0}


% tightlist command for lists without linebreak
\providecommand{\tightlist}{%
  \setlength{\itemsep}{0pt}\setlength{\parskip}{0pt}}

% From pandoc table feature
\usepackage{longtable,booktabs,array}
\usepackage{calc} % for calculating minipage widths
% Correct order of tables after \paragraph or \subparagraph
\usepackage{etoolbox}
\makeatletter
\patchcmd\longtable{\par}{\if@noskipsec\mbox{}\fi\par}{}{}
\makeatother
% Allow footnotes in longtable head/foot
\IfFileExists{footnotehyper.sty}{\usepackage{footnotehyper}}{\usepackage{footnote}}
\makesavenoteenv{longtable}



\usepackage{setspace}
\doublespacing        % or \onehalfspacing for 1.5 lines



\begin{document}


\begin{frontmatter}

  \title{Agroforestry SmartAgroforst: Poplar biomass modelling}
    \author[1,2]{Thomas Kirschstein%
  %
  }
   \ead{thomas.kirschstein@ikts.fraunhofer.de} 
      \cortext[cor1]{Corresponding author}
  
  \begin{abstract}
  
  \end{abstract}
  
 \end{frontmatter}

\section{Abstract}\label{abstract}

Short-rotation agroforestry systems are increasingly promoted as a
low-carbon feedstock option for bio-based value chains, but their
economic viability depends on coordinated decisions on stand
establishment, multi-cycle harvesting, and product cascading in
spatially distributed supply chains. This paper develops a mixed-integer
linear programming (MILP) model for the integrated design and operation
of an agroforestry biomass supply chain with explicit age-lag
constraints that link consecutive harvests at each site. The formulation
couples allometric yield functions for different biomass types with a
multi-product, multi-period network design model including storage,
processing, and product quality cascades. The model is applied to a
stylised case study of Central European poplar-based alley-cropping
systems to explore trade-offs between harvest rotation length, product
portfolios, and infrastructure configuration. We outline a computational
experimentation plan to test the impact of allometric parameter
uncertainty, price scenarios, and policy incentives on the optimal
deployment of agroforestry systems in cleaner bioeconomy pathways.

\textbf{Keywords:} Agroforestry; Short-rotation coppice; Biomass supply
chain; Allometric models; Mixed-integer linear programming; Product
cascading

\section{Introduction}\label{introduction}

Agroforestry systems that integrate trees with agricultural crops are
increasingly recognised as a promising land-use option to reconcile
climate mitigation, biodiversity conservation, and rural income
generation on the same hectare of land
\citep{Nair1993_IntroAgroforestry, Toensmeier2017_Carbon_Agroforestry}.
Temperate short-rotation agroforestry systems (SRAFS) based on
fast-growing species such as poplar, willow, alder, and black locust can
deliver competitive biomass yields while improving soil quality and
providing multiple ecosystem services compared to conventional
monocultures \citep{Huber2018_SRAFS_growth, Trnka2008_Poplar_SRF}. At
the same time, the deployment of wood-based bioenergy and bioproduct
value chains raises concerns regarding land competition, lifecycle
emissions, and economic risks, which calls for integrative planning
tools that capture both landscape dynamics and supply chain performance
\citep{Zahraee2020_BSC_review, Espinoza2017_Biorefinery_SC_review}.

In the last two decades, a large body of research has analysed biomass
supply chains for bioenergy and biorefineries using mathematical
programming models, often in the form of mixed-integer linear
programming (MILP)
\citep{Zahraee2020_BSC_review, Sharma2013_BSC_review}. These models
typically optimise network configuration, feedstock sourcing, and
logistics under cost or profit objectives, but they often treat biomass
availability as an exogenous input or rely on simplified yield functions
that neglect stand growth dynamics and age-related constraints. Recent
work has started to integrate biomass growth and regeneration processes
in multi-period supply chain models, yet most applications focus on
grassland or forest residues rather than agroforestry systems with
repeated coppice cycles and strong interactions between stand structure
and harvesting decisions \citep{DeMeyer2016_tOPTIMASS}.

Short-rotation agroforestry systems exhibit pronounced temporal
variability in biomass production due to species-specific growth
trajectories, competition, and management practices, which has been
documented in long-term trials in Central Europe
\citep{Huber2018_SRAFS_growth, Trnka2008_Poplar_SRF}. For poplar,
willow, alder, and black locust, allometric tree biomass models based on
stem base diameter (SBD) or similar biometric variables provide a
compact and empirically grounded link between stand structure and
aboveground woody biomass at the tree and stand level
\citep{Huber2016_Allometric_SRAFS}. These models allow the derivation of
age-dependent yield coefficients for different product types, but they
are rarely coupled with tactical and strategic supply chain
optimisation. From a cleaner production perspective, it is particularly
relevant to capture the cascading use of biomass along quality gradients
(e.g., chemical feedstocks, pulp, and energy), as this can significantly
affect both economic returns and environmental outcomes
\citep{DeMeyer2015_OPTIMASS, Lamers2015_Feedstock_Design}.

The present paper contributes to this literature in four ways. First, we
propose a novel MILP formulation for agroforestry biomass supply chain
design that explicitly links consecutive harvest cycles through age-lag
constraints and binary path variables, ensuring that harvesting
decisions at each site are consistent with minimum and maximum rotation
ages over multiple cycles. Second, we embed allometric yield functions
for different biomass types into the model by translating age-specific
biomass increments into site- and product-specific yield parameters,
thereby capturing species- and management-dependent growth dynamics in
an optimisation-compatible way
\citep{Huber2016_Allometric_SRAFS, Trnka2008_Poplar_SRF}. Third, we
model a product cascading hierarchy in which higher-quality products can
satisfy lower-quality demands, which is relevant for cleaner production
strategies that prioritise material uses before energy recovery
\citep{DeMeyer2015_OPTIMASS, Lamers2015_Feedstock_Design}. Finally, we
outline a computational experimentation framework for a regional case
study in Central Europe that will test the sensitivity of optimal system
design to rotation length, allometric parameter uncertainty, price and
policy scenarios, and infrastructure options.

The remainder of the paper is structured as follows. Section
@ref(sec:litreview) reviews the relevant literature on agroforestry
biomass systems, allometric biomass modelling, and biomass supply chain
optimisation. Section @ref(sec:allometrics) introduces the allometric
models adopted to represent biomass growth of different tree species and
biomass types. Section @ref(sec:milp) summarises the MILP formulation
for the agroforestry supply chain with age-lag constraints. Section
@ref(sec:experiments) outlines the planned computational experiments for
the case study and discusses expected insights for cleaner production
strategies.

\section{Literature review}\label{sec:litreview}

\subsection{Agroforestry and short-rotation
systems}\label{agroforestry-and-short-rotation-systems}

Agroforestry encompasses a wide range of land-use systems in which woody
perennials are deliberately integrated with crops and/or livestock on
the same land management unit \citep{Nair1993_IntroAgroforestry}. In
temperate regions, short-rotation agroforestry systems (SRAFS) often
take the form of alley-cropping, where double or multiple rows of
fast-growing trees alternate with crop strips, enabling simultaneous
production of agricultural commodities and woody biomass
\citep{Huber2018_SRAFS_growth}. Empirical studies in Southern Germany
and Central Europe have shown that SRAFS based on poplar, willow, black
alder, and black locust can achieve mean annual increments (MAI) of
about 7--10 t ha\(^{-1}\) a\(^{-1}\) in aboveground woody biomass during
the first rotation, with species-specific differences in growth
dynamics, size hierarchies, and mortality
\citep{Huber2018_SRAFS_growth, Trnka2008_Poplar_SRF}.

\citet{Huber2018_SRAFS_growth} analyse the growth dynamics and stand
structure development of four tree species in organic and conventional
SRAFS over a four-year rotation in Southern Germany. They report that
poplar and black locust exhibit the highest mean annual biomass
increments (around 10 t ha\(^{-1}\) a\(^{-1}\) over four years), while
willow shows lower biomass due to lower diameter growth and wood
density, despite high shoot numbers. \citet{Trnka2008_Poplar_SRF} report
similar MAI ranges (10--15 t ha\(^{-1}\) a\(^{-1}\)) for short-rotation
poplar clones under high-density plantations in the Czech-Moravian
highlands, highlighting the importance of clone selection and site
conditions. These findings underline that species choice, rotation
length, planting density, and management practices (e.g., coppicing,
weed control) crucially influence biomass yields and thus the design of
downstream supply chains.

Economic assessments of short-rotation coppice (SRC) and very short
rotation coppice (vSRC) poplar plantations in Italy and Central Europe
show that SRC systems with 5--7 year rotations and moderate planting
densities can be financially competitive with conventional arable crops,
particularly when biomass prices and policy support are favourable
\citep{Testa2014_Poplar_Economics}. However, profitability is sensitive
to yield assumptions, establishment costs, and logistics, motivating
integrated analyses that couple stand-level growth models with supply
chain optimisation. Beyond yield and economics, agroforestry and SRC
systems are widely studied for their environmental performance,
including carbon sequestration, nutrient cycling, biodiversity, and soil
and water protection
\citep{Toensmeier2017_Carbon_Agroforestry, Huber2018_SRAFS_growth}.
Cleaner production strategies that leverage agroforestry biomass in
biorefinery and bioenergy systems therefore require tools that can
simultaneously account for temporal biomass availability, quality
differentiation, and value chain design.

\subsection{Allometric models for tree
biomass}\label{allometric-models-for-tree-biomass}

Allometric tree biomass models relate easily measurable biometric
variables such as diameter and height to aboveground biomass, typically
using power-law relationships of the form \[
M = b_0 \cdot \text{SBD}^{\,b_1},
\] where \(M\) is the oven-dry biomass of an individual tree, SBD is
stem base diameter, and \(b_0\) and \(b_1\) are species-specific
coefficients \citep{Huber2016_Allometric_SRAFS}. Such models are
attractive because they allow the estimation of tree and stand biomass
from non-destructive measurements and can be embedded in growth and
yield models for management and planning purposes
\citep{Huber2018_SRAFS_growth}.

For SRAFS in Southern Germany, \citet{Huber2016_Allometric_SRAFS} and
\citet{Huber2018_SRAFS_growth} develop species-specific allometric
models based on SBD measured 10 cm above ground to predict aboveground
leafless dry biomass of black alder, black locust, poplar (Max 3), and
willow (Inger). They report common exponents \(b_1 = 2.603\) across
species, with species-specific factors \(b_0\) of 0.025, 0.041, 0.036,
and 0.037, respectively, reflecting differences in wood density and
branching patterns \citep{Huber2016_Allometric_SRAFS}. These models,
calibrated on destructive sampling at the same site, are used to
estimate annual biomass increments and derive MAI and current annual
increment (CAI) for each species and management system
\citep{Huber2018_SRAFS_growth}. The study emphasises that diameter
distributions, stand inequality, and mortality strongly affect aggregate
yields, particularly for black locust, which shows large size asymmetry
and high mortality, and for willow, where late sprouting leads to
bimodal size distributions \citep{Huber2018_SRAFS_growth}.

Similar allometric approaches have been applied to SRC and SRF poplar
plantations. \citet{Trnka2008_Poplar_SRF} use diameter and height
measurements to estimate aboveground biomass and assess clone-specific
biomass production and survival over a six-year rotation in the Czech
Republic. Laureysens and co-authors provide allometric equations for
poplar and willow clones on waste disposal sites and agricultural land,
often highlighting the role of wood density, site conditions, and stand
structure for biomass estimates. In agroforestry contexts, allometric
models have also been developed for mixed-species systems and
silvopastoral systems, which can be used to derive carbon stock
estimates and evaluate management options
\citep{Toensmeier2017_Carbon_Agroforestry}.

From a modelling perspective, allometric functions can be used in two
main ways when coupling stand growth with supply chain optimisation.
First, they can generate age-specific yield tables (e.g., t ha\(^{-1}\)
at each stand age) that serve as exogenous parameters in linear or
integer programming models. Second, they can be integrated more
dynamically into optimisation models through piecewise linear
approximations or age-class state variables. For large-scale MILP
models, the first approach is often preferred due to computational
tractability and the linearity requirement, but it needs to capture key
nonlinearities such as the timing of MAI peaks and constraints on
minimum and maximum harvest ages.

\subsection{Age-dependent biomass yield
functions}\label{age-dependent-biomass-yield-functions}

Existing allometric models link tree biomass to diameter or height but
do not yet provide \textbf{stand-level} biomass as an explicit function
of stand age. Age-dependent biomass functions are useful for yield
projections in short-rotation forestry and agroforestry, where
management is organized in discrete rotations (e.g.~5--15 years).
\citep{Niemczyk2021, Testa2014}

A common approach is to derive age functions from rotation trials that
report mean annual increment (MAI) over time and identify the age at
which MAI plateaus. For poplar SRWC in northern Poland, Niemczyk et
al.~(2021) showed that MAI in aboveground dry biomass increases up to
about age 10; beyond this, the MAI curve flattens, suggesting a
biologically optimal rotation age around 10 years because the slope of
MAI is near zero. \citep{Niemczyk2021} Reported MAI values for
single-stem poplar plantations in a 10-year rotation ranged
approximately from 1--15 Mg DM ha\(^{-1}\) yr\(^{-1}\) depending on
cultivar. \citep{Niemczyk2021} Similar magnitudes and rotation lengths
(about 5--7 years for SRC, up to 10--15 years for SRC/SRF) are reported
for poplar SRC systems in Italy, where whole-rotation averages of
roughly 15 Mg DM ha\(^{-1}\) yr\(^{-1}\) are considered economically
viable. \citep{Testa2014}

A simple and widely used parametric form for age-dependent stand biomass
\(B(t)\) is the Chapman--Richards (or Mitscherlich/monomolecular)
function: \[
B(t) = B_{\max}\left(1 - e^{-k\,t}\right)^m
\] with stand age \(t\) in years, asymptotic biomass \(B_{\max}\) (Mg DM
ha\(^{-1}\)), rate parameter \(k\) (year\(^{-1}\)), and shape parameter
\(m\). This formulation captures the rapid juvenile growth, a phase of
approximately linear biomass accumulation, and a saturating phase where
growth slows. \citep{Niemczyk2021} The corresponding current annual
increment (CAI) and mean annual increment (MAI) are \[
\text{CAI}(t) = \frac{dB}{dt}, \qquad \text{MAI}(t) = \frac{B(t)}{t}.
\] The age maximizing MAI (optimal rotation under a volume/biomass
criterion) can be derived analytically or determined numerically.
Fitting such functions to observed age--biomass or age--volume series
from rotation trials (e.g.~Niemczyk et al.~2021 for poplar, Testa et
al.~2014 for Italian SRC) allows cultivar-specific yield curves that are
directly usable in agroforestry simulation models.
\citep{Niemczyk2021, Testa2014}

For illustration, we parameterize three stylized poplar curves
representing low, medium, and high productivity on temperate mineral
soils. Parameter values are loosely informed by the ranges reported by
Niemczyk et al.~(2021) and Testa et al.~(2014) and should be
recalibrated when site-specific data are available.
\citep{Niemczyk2021, Testa2014}

\begin{itemize}
\tightlist
\item
  Low productivity: \(B_{\max} = 80\) Mg DM ha\(^{-1}\), \(k = 0.18\),
  \(m = 1.1\)
\item
  Medium productivity: \(B_{\max} = 120\) Mg DM ha\(^{-1}\),
  \(k = 0.20\), \(m = 1.2\)
\item
  High productivity: \(B_{\max} = 160\) Mg DM ha\(^{-1}\), \(k = 0.22\),
  \(m = 1.3\)
\end{itemize}

These values yield MAI maxima between about 8 and 12 years with peak MAI
values of roughly 8--16 Mg DM ha\(^{-1}\) yr\(^{-1}\), consistent with
the empirical range where 10-year rotations produced MAI of about 10--15
Mg DM ha\(^{-1}\) yr\(^{-1}\) for the best poplar cultivars in northern
Poland and Italy. \citep{Niemczyk2021, Testa2014}

In the agroforestry context, such age functions can be combined with
tree density and allometric partitioning to above- and below-ground
biomass to obtain tree-row biomass over time for different designs.
Calibration against local inventory data or growth measurements
(e.g.~basal area increment and height growth) is recommended before
using the curves for scenario analysis.

\begin{figure}
\includegraphics[width=1\linewidth]{paper_files/figure-latex/age-biomass-functions-1} \caption{Age-dependent stand biomass (a) and mean annual increment (b) for three stylized poplar productivity levels.}\label{fig:age-biomass-functions}
\end{figure}

\subsection{Biomass supply chain modelling and MILP
approaches}\label{biomass-supply-chain-modelling-and-milp-approaches}

Biomass supply chain design and planning has been widely studied using
MILP models at strategic, tactical, and operational levels
\citep{Zahraee2020_BSC_review, Sharma2013_BSC_review}. Comprehensive
reviews classify models according to decision scope (location, capacity,
routing, inventory), time horizon, uncertainty treatment, and
sustainability metrics
\citep{Zahraee2020_BSC_review, Espinoza2017_Biorefinery_SC_review}. Many
models focus on bioenergy supply chains for electricity, heat, or
biofuels, considering feedstocks such as agricultural residues, forestry
residues, dedicated energy crops, and municipal wastes
\citep{Sharma2013_BSC_review}. Environmental aspects, including
greenhouse gas emissions and land use, are increasingly integrated,
often using life-cycle assessment (LCA) data or emission factors
\citep{Espinoza2017_Biorefinery_SC_review}.

Among generic models, OPTIMASS is a notable example of a multi-product
MILP that optimises biomass supply chains with changing biomass
characteristics and by-product re-injection
\citep{DeMeyer2015_OPTIMASS}. It models the flow of multiple biomass
types and products through supply chain stages, capturing conversion
yields, storage, and cascading uses. The t-OPTIMASS extension further
incorporates biomass growth and regeneration over multiple periods,
enabling the representation of time-dependent availability for biomass
supply chains \citep{DeMeyer2016_tOPTIMASS}. However, these formulations
generally assume exogenous biomass availability profiles per period, and
do not explicitly model stand-level age dynamics or rotation
constraints.

Several scenario-based approaches have been developed for forest
biorefinery supply chains, linking product portfolios, process choices,
and supply chain design under market volatility
\citep{Mansoornejad2013_Forest_Biorefinery_SC}.
\citet{Lamers2015_Feedstock_Design} discuss strategic feedstock supply
system design for biorefineries, emphasising the importance of economies
of scale, conversion yields, and logistics in achieving cost-efficient
and robust systems. Reviews highlight that only a limited subset of
models explicitly consider biomass growth, spatially explicit land-use
change, or agroforestry systems
\citep{Zahraee2020_BSC_review, Sharma2013_BSC_review}. There is
therefore a clear opportunity to develop models that couple agroforestry
stand dynamics with network design and cleaner production objectives.

\subsection{Cleaner production and cascading use of
biomass}\label{cleaner-production-and-cascading-use-of-biomass}

From a cleaner production perspective, the cascading use of
biomass---prioritising high-value material uses before energy
recovery---is widely advocated to maximise resource efficiency and
mitigate environmental impacts \citep{Lamers2015_Feedstock_Design}. In
wood-based value chains, this implies that high-quality timber or
chemical feedstock fractions should first be allocated to material
products such as engineered wood products, pulp and paper, or
biochemicals, while lower-quality fractions and residues are used for
energy \citep{DeMeyer2015_OPTIMASS, Lamers2015_Feedstock_Design}. MILP
models can represent such cascades by defining product quality classes
and allowing substitution of higher-quality products for lower-quality
demands, subject to hierarchy constraints.

The OPTIMASS framework explicitly models such product cascades by
defining a product quality order and allowing flows from higher-quality
products to lower-quality demand nodes \citep{DeMeyer2015_OPTIMASS}.
Similar ideas have been applied in forest-based biorefinery supply
chains, where product portfolios include platform chemicals, fuels, and
electricity \citep{Mansoornejad2013_Forest_Biorefinery_SC}. For
agroforestry biomass, poplar and willow can supply different product
streams, including sawlogs, peeled veneers, pulpwood, and energy chips,
with quality primarily defined by diameter, straightness, and defect
rates \citep{Testa2014_Poplar_Economics, Spinelli2009_SRC_Harvesting}.
Economic analyses in Italy and Central Europe suggest that the economic
viability of SRC systems depends on access to higher-value markets such
as panel products and pulp, in addition to energy markets
\citep{Testa2014_Poplar_Economics}.

Cleaner production strategies that integrate agroforestry biomass into
regional bioeconomies therefore require models that can allocate biomass
flows along such cascades while respecting stand growth dynamics,
rotation constraints, and logistical capacities. Explicit representation
of age-dependent yield quality (e.g., diameter distributions, wood
density) remains challenging in large-scale MILP models, but can be
approximated through age- and species-specific product yield fractions.

\section{Allometric models for biomass growth}\label{sec:allometrics}

\subsection{Tree-level allometry in short-rotation
agroforestry}\label{tree-level-allometry-in-short-rotation-agroforestry}

In the SRAFS trial in Southern Germany,
\citet{Huber2016_Allometric_SRAFS} estimate aboveground leafless dry
biomass of individual trees using allometric functions with SBD as
predictor, derived from destructive sampling at the same site. The
general functional form is \[
M_{ij} = b_{0,j} \cdot \text{SBD}_{i}^{\,b_{1,j}},
\] where \(M_{ij}\) is the aboveground dry biomass (kg) of tree \(i\) of
species \(j\), \(\text{SBD}_{i}\) is stem base diameter (cm) at 10 cm
above ground, and \(b_{0,j}\), \(b_{1,j}\) are species-specific
parameters \citep{Huber2016_Allometric_SRAFS}. The coefficients reported
for the four species are summarised in Table @ref(tab:allometrics). All
biomass values are expressed as oven-dry mass
\citep{Huber2016_Allometric_SRAFS}.

\begin{longtable}[]{@{}lrr@{}}
\caption{Allometric coefficients for aboveground biomass of tree species
in SRAFS \citep{Huber2016_Allometric_SRAFS}.
\{\#tab:allometrics\}}\tabularnewline
\toprule\noalign{}
Species & \(b_0\) & \(b_1\) \\
\midrule\noalign{}
\endfirsthead
\toprule\noalign{}
Species & \(b_0\) & \(b_1\) \\
\midrule\noalign{}
\endhead
\bottomrule\noalign{}
\endlastfoot
Black alder (\emph{Alnus glutinosa}) & 0.025 & 2.603 \\
Black locust (\emph{Robinia pseudoacacia}) & 0.041 & 2.603 \\
Poplar Max 3 (\emph{Populus} hybrid) & 0.036 & 2.603 \\
Willow Inger (\emph{Salix} hybrid) & 0.037 & 2.603 \\
\end{longtable}

These functions are applied to all shoots per tree to compute total tree
biomass, which is then aggregated at plot and stand level (t
ha\(^{-1}\)) using observed tree densities
\citep{Huber2018_SRAFS_growth}. By repeating measurements annually over
four years, the authors derive CAI and MAI for each species and
management system, and analyse the evolution of size distributions, Gini
coefficients, and skewness as indicators of stand structure
\citep{Huber2018_SRAFS_growth}. Their results show that black locust
exhibits high wood density (around 0.60 g cm\(^{-3}\)) and some
large-diameter trees, leading to biomass comparable to poplar despite
lower mean diameter and height \citep{Huber2018_SRAFS_growth}. Willow,
in contrast, shows low wood density and lower diameter growth, resulting
in lower biomass despite vigorous sprouting and high shoot densities
\citep{Huber2018_SRAFS_growth}.

\subsection{Stand-level yield functions and age
classes}\label{stand-level-yield-functions-and-age-classes}

To integrate allometric growth into a MILP formulation, we abstract from
individual trees and construct age-class yield functions for each
biomass type \(p\) and age \(a\). Following common practice in forest
and SRC modelling, we define discrete annual age classes and compute the
stand biomass yield \(Y_{pa}\) (t ha\(^{-1}\)) associated with
harvesting at age \(a\) by combining allometric functions with
empirically observed diameter distributions and stand densities
\citep{Huber2018_SRAFS_growth, Trnka2008_Poplar_SRF}. Where detailed
stand structure data are available, this can be done by simulating SBD
distributions over time (e.g., via mixed-effects models) and applying
tree-level allometrics to each diameter class
\citep{Huber2018_SRAFS_growth}. In the absence of such detailed data,
age-dependent MAI and CAI curves reported in the literature can be used
to approximate \(Y_{pa}\)
\citep{Trnka2008_Poplar_SRF, Huber2018_SRAFS_growth}.

In the context of the agroforestry supply chain design (SCD) model, we
assume that each agroforestry site \(i\) is characterised by a species
or species mix, a planting density, and management regime, which jointly
determine a site-specific yield profile \(\eta_{pa}\) (t ha\(^{-1}\))
for each product type \(p\) at age \(a\). These yield profiles can be
interpreted as the maximum harvestable biomass per hectare for product
\(p\) if the stand is harvested at age \(a\), conditional on the stand
having been established at an earlier time. In the current MILP
formulation, these profiles are captured through the parameter
\(\eta_{pa}\), termed base yield, and are linked to decision variables
via the constraint \[
Y_{ipt} = \sum_{s=1}^{t-1} \eta_{p(t-s)} \cdot \text{AREA}_i \cdot z_{ist} \quad \forall i,p,t \in \mathcal{T}^{\text{harv}}.
\] Here, \(Y_{ipt}\) is the maximum harvest quantity of product \(p\) at
site \(i\) in period \(t\), \(\text{AREA}_i\) is the site area (ha), and
\(z_{ist}\) is a binary variable indicating that harvests at site \(i\)
in periods \(s\) and \(t\) form consecutive harvests with an age lag
\(t-s\) within allowed bounds. The term \(\eta_{p(t-s)}\) thus plays the
role of an allometrically derived yield coefficient that depends only on
stand age \(a = t-s\), not on calendar time.

\subsection{Biomass types and product quality
classes}\label{biomass-types-and-product-quality-classes}

In the model, three product types are distinguished: \(p = 1\)
(chemical-grade biomass), \(p = 2\) (pulp-grade biomass), and \(p = 3\)
(energy-grade biomass). This tripartite classification reflects a
product quality hierarchy, with product 1 having the highest quality and
price, and product 3 being the lowest-quality fraction typically used
for energy \citep{DeMeyer2015_OPTIMASS, Lamers2015_Feedstock_Design}.
From an allometric perspective, the differentiation among these products
can be linked to diameter classes, stem straightness, and defect rates,
as these determine the suitability of biomass fractions for different
processing routes
\citep{Testa2014_Poplar_Economics, Spinelli2009_SRC_Harvesting}.

In practice, allometric models provide estimates of total aboveground
woody biomass, which can be split into product classes using empirical
conversion factors or allocation rules derived from product recovery
studies (e.g., share of biomass suitable for veneer vs.~pulp vs.~chips
as a function of diameter and log length)
\citep{Spinelli2009_SRC_Harvesting}. For the purposes of the SCD model,
we can represent this by specifying product-specific yield coefficients
\(\eta_{pa}\) that sum to the total biomass yield at age \(a\) and
reflect the expected proportion of each product class under a given
management regime.

The cascading hierarchy in the MILP formulation is implemented through
demand satisfaction constraints that allow higher-quality products to
satisfy lower-quality demands. Specifically, demand \(D_{kpt}^{\max}\)
at consumer \(k\) for product class \(p\) in period \(t\) can be met by
flows of product \(p\) and all higher-quality classes \(p' \leq p\) from
storage hubs, as captured by \[
\sum_{j \in \mathcal{J}, p' \in \mathcal{P}: p' \leq p } X_{jkp'pt} \leq D_{kpt}^{\max} \quad \forall k,p,t.
\] This structure supports cleaner production by favouring material uses
when possible and relegating residual fractions to energy uses,
depending on revenue parameters \(R_{k,p}\) and system constraints
\citep{DeMeyer2015_OPTIMASS, Lamers2015_Feedstock_Design}.

\section{MILP model for agroforestry supply chain
design}\label{sec:milp}

\subsection{Sets, indices, and time
structure}\label{sets-indices-and-time-structure}

The model considers a set of agroforestry sites \(\mathcal{I}\),
storage/processing facilities \(\mathcal{J}\), consumer sites
\(\mathcal{K}\), and product types \(\mathcal{P} = \{1,2,3\}\). Time is
discretised into annual periods
\(t \in \mathcal{T} = \{1,\dots,T_{\max}\}\). To represent stand
establishment and harvest timing, extended time sets
\(\mathcal{T}^{+} = \{0,\dots,T_{\max}+1\}\) and harvest periods
\(\mathcal{T}^{\text{harv}} = \{A^{\min}+1,\dots,T_{\max}\}\) are
defined, where \(A^{\min}\) and \(A^{\max}\) are minimum and maximum
stand ages (years) at harvest. The set \(\mathcal{S}\) comprises all
ordered pairs \((s,t)\) of consecutive harvest periods (including
establishment at \(s=0\)) satisfying the age-lag bounds: \[
\mathcal{S} = \{(s,t) \mid s,t \in \{0,\dots,T_{\max}+1\},\; t-s \in \{A^{\min},\dots,A^{\max}\} \}.
\]

\subsection{Decision variables}\label{decision-variables}

Binary variables \(z_{ist}\) indicate whether, at site \(i\), periods
\(s\) and \(t\) form consecutive harvests (or establishment and first
harvest if \(s=0\)), with \((s,t) \in \mathcal{S}\). These variables
define a path of consecutive harvest cycles at each site over the
planning horizon. Continuous variables \(Y_{ipt} \ge 0\) represent the
maximum harvest quantity (t) of product \(p\) at site \(i\) in harvest
period \(t\). Flow variables \(X_{ijpt}\) and \(X_{jkpp't}\) represent
transport of products from sites to storage and from storage to
consumers, respectively, and \(S_{jpt}\) represent inventories at
storage locations.

\subsection{Objective function}\label{objective-function}

The model maximises net present value over the planning horizon,
expressed as total revenues minus establishment, opportunity, harvest,
transport, and storage costs: \[
\begin{aligned}
\max Z =\;&
\sum_{k,p,t} R_{k,p} \cdot \sum_{j \in \mathcal{J},\, p' \in \mathcal{P}: p' \leq p} X_{jkp'pt} \\
& - \sum_{i} C^{\text{est}}_{i} \cdot \text{AREA}_i \cdot \sum_{t \in \mathcal{T}} z_{i0t} \\
& - \sum_{i} C^{\text{opp}}_{i} \cdot \text{AREA}_i \cdot \sum_{t \in \mathcal{T}} (T_{\max}-t) \cdot z_{i0t} \\
& - \sum_{i,(s,t)\in \mathcal{S}: s>0} C^{\text{harv}}_{i} \cdot \text{AREA}_i \cdot z_{ist} \\
& - c^{\text{tr-raw}} \sum_{i,j,p,t} d_{ij} \cdot X_{ijpt}
- c^{\text{tr-pre}} \sum_{j,k,p,t} d_{jk} \cdot X_{jkpt}
- \sum_{j,p,t} c^{\text{stor}}_{j} \cdot S_{jpt}.
\end{aligned}
\]

Here, \(R_{k,p}\) are product- and consumer-specific revenues (€/t),
\(C^{\text{est}}_{i}\) are establishment costs per hectare,
\(C^{\text{opp}}_{i}\) are annual opportunity costs per hectare, and
\(C^{\text{harv}}_{i}\) are harvest and refitting costs per hectare
\citep{Testa2014_Poplar_Economics}. Transport costs are proportional to
flow and distance with rates \(c^{\text{tr-raw}}\) and
\(c^{\text{tr-pre}}\) for raw and pre-treated biomass, respectively, and
storage incurs linear holding costs \(c^{\text{stor}}_j\) per tonne.

\subsection{Key constraints}\label{key-constraints}

The path connectivity constraint ensures that each harvest period at a
site has exactly one predecessor and one successor in the age-lag graph,
effectively forming a path over \(\mathcal{T}^{+}\): \[
\sum_{s=0: (s,t)\in \mathcal{S}}^{t} z_{ist} = \sum_{u = t+1: (t,u) \in \mathcal{S}}^{T_{\max}+1} z_{itu}
\quad \forall i \in \mathcal{I}, t \in \mathcal{T}.
\]

Establishment is restricted to at most one time per site: \[
\sum_{t \in \mathcal{T}^{+}} z_{i0t} \le 1 \quad \forall i \in \mathcal{I}.
\]

Biomass yield is linked to age-lag decisions through the allometric
yield coefficients: \[
Y_{ipt} = \sum_{s=1}^{t-1} \eta_{p(t-s)} \cdot \text{AREA}_i \cdot z_{ist}
\quad \forall i \in \mathcal{I}, p \in \mathcal{P}, t \in \mathcal{T}^{\text{harv}}.
\]

Harvest quantities bound outbound flows from each site: \[
\sum_{j \in \mathcal{J}} X_{ijpt} \le Y_{ipt}
\quad \forall i,p,t \in \mathcal{T}^{\text{harv}}.
\]

At storage hubs, inventories obey standard balance equations with
initial inventory zero: \[
S_{jpt} = S_{jp,t-1} + \sum_{i} X_{ijpt} - \sum_{k, p' \ge p} X_{jkpp't}
\quad \forall j,p,t \in \mathcal{T}^{\text{harv}}.
\]

Storage and processing capacities are enforced via: \[
\sum_{p} S_{jpt} \le \text{CAP}^{\text{stor}}_j \quad \forall j,t, \qquad
\sum_{i,p} X_{ijpt} \le \text{CAP}^{\text{proc}}_j \quad \forall j,t.
\]

Demand satisfaction with cascades is modelled as described above, where
higher-quality products can satisfy lower-quality demands up to maximum
demand levels \(D^{\max}_{kpt}\).

\section{Case study and computational experiment
plan}\label{sec:experiments}

The proposed MILP model is intended to support the design of
agroforestry biomass supply chains for cleaner production in specific
regional contexts. In a forthcoming case study, we focus on poplar-based
alley-cropping systems in a temperate European setting, informed by
empirical data from SRAFS trials in Southern Germany and SRC plantations
in Central Europe
\citep{Huber2018_SRAFS_growth, Trnka2008_Poplar_SRF, Testa2014_Poplar_Economics}.

We envisage the following steps and experimental scenarios for
computational analysis:

\begin{enumerate}
\def\labelenumi{\arabic{enumi}.}
\item
  \textbf{Parameterisation of allometric yield profiles:} Use species-
  and site-specific allometric models and observed growth trajectories
  to construct age-dependent yield coefficients \(\eta_{pa}\) for
  chemical, pulp, and energy product classes at representative
  agroforestry sites. Where necessary, scenario ranges will be defined
  to capture uncertainty in MAI and CAI due to climate variability and
  management differences
  \citep{Huber2016_Allometric_SRAFS, Huber2018_SRAFS_growth, Trnka2008_Poplar_SRF}.
\item
  \textbf{Baseline optimisation:} Solve the MILP model for a baseline
  configuration with fixed prices, capacities, and policy settings,
  obtaining optimal establishment timing, harvest schedules, and supply
  chain structure (locations, flows, inventories).
\item
  \textbf{Rotation length scenarios:} Vary the age-lag bounds
  \(A^{\min}\) and \(A^{\max}\) to represent alternative rotation
  strategies (e.g., 3--5 years vs.~8--10 years) and analyse their
  impacts on profitability, product mix, and land use. This will shed
  light on trade-offs between shorter rotations favouring energy chips
  and longer rotations enabling higher shares of material products
  \citep{Trnka2008_Poplar_SRF, Testa2014_Poplar_Economics}.
\item
  \textbf{Product price and cascading scenarios:} Explore scenarios with
  different relative prices for chemical, pulp, and energy products,
  reflecting changes in market demand and policy incentives (e.g.,
  bioenergy subsidies, green chemicals markets). Assess how the product
  cascade hierarchy and yield quality structure influence the marginal
  value of species and rotation decisions
  \citep{DeMeyer2015_OPTIMASS, Lamers2015_Feedstock_Design}.
\item
  \textbf{Infrastructure and logistics scenarios:} Evaluate the effect
  of alternative storage and processing network configurations (e.g.,
  decentralised vs.~centralised hubs, different capacity levels,
  transport cost parameters) on the optimal spatial deployment of
  agroforestry sites and biomass flows
  \citep{DeMeyer2015_OPTIMASS, DeMeyer2016_tOPTIMASS, Lamers2015_Feedstock_Design}.
\item
  \textbf{Sensitivity to allometric and growth uncertainty:} Conduct
  parametric sensitivity and, where computationally feasible, stochastic
  analyses to quantify the impact of uncertainty in allometric
  parameters and growth responses (e.g., under drought or management
  differences) on optimal decisions and system robustness
  \citep{Huber2018_SRAFS_growth, Trnka2008_Poplar_SRF}.
\item
  \textbf{Cleaner production indicators:} For selected solutions,
  compute derived indicators such as biomass cascading index (share of
  biomass used for material products before energy), utilisation rates
  of higher-quality fractions, and, where coupled with LCA data,
  greenhouse gas emission savings relative to fossil benchmarks
  \citep{Espinoza2017_Biorefinery_SC_review, Zahraee2020_BSC_review}.
\end{enumerate}

\section{Conclusions}\label{conclusions}

This paper outlines a research framework that couples allometric biomass
growth models with a MILP-based agroforestry biomass supply chain design
model featuring age-lag constraints and product cascading. Building on
empirical evidence from temperate SRAFS and SRC systems and established
biomass supply chain models, the proposed approach aims to capture key
interactions between stand-level growth dynamics and network-level
design decisions that are central to the cleaner production potential of
agroforestry
\citep{Huber2018_SRAFS_growth, DeMeyer2015_OPTIMASS, DeMeyer2016_tOPTIMASS}.
Future work will implement and calibrate the model for a regional case
study, conduct the computational experiments sketched above, and extend
the framework towards multi-objective optimisation including
environmental indicators such as greenhouse gas emissions and
biodiversity proxies.

\bibliography{references.bib}


\end{document}
